\documentclass[12pt, a4paper]{article}
\usepackage[UTF8, scheme = plain]{ctex}
\renewcommand\thesection{\Alph{section}} \renewcommand\thesubsection{\thesection.\arabic{subsection}}
\usepackage{geometry}
\geometry{left=1.5cm,right=2cm,top=2cm,bottom=1.5cm}

% Title
\title{暑期集训第一次积分训练赛}
\author{河南理工大学ACM协会 \thanks{name1 name2}}
\date{2018/7/13}

\begin{document}

\maketitle\newpage

% *************************************      仅修改以下内容        *******************************************

\section{辞树的肥宅快乐水}

\begin{table}[!h]
  \centering
  \begin{tabular}{l|l|l}
  时间限制 & 内存限制 & 出题人 \\
  \hline
  1 Second & 512 Mb & 吴丰源 \\
\end{tabular}
\end{table}

\subsection*{题目描述}

又到了基情四射的夏天,大家出去约妹子,而肥宅辞树只想宅在机房喝肥宅快乐水。
辞树一下子买了n瓶肥宅快乐水。已知他一天里至少喝掉一瓶肥宅水且他是一口干掉一整瓶。(肥宅Orz)
他想要知道自己一共有多少种喝肥宅水的方案。
两种方案被认为是不同的,当且仅当辞树买的这些肥宅水能喝的天数不同,或者存在一天两种方案喝的肥宅水瓶数不同。

\subsection*{输入}

第一行输入一个正整数T,代表有T组数据$(0<T<11)$
每组数据有一个正整数n,代表辞树买了n瓶肥宅快乐水。$(0<n<10^8)$

\subsection*{输出}

对于每组数据,输出一行,将方案数用二进制表示输出。

\subsection*{输入样例}

1\newline
3

\subsection*{输出样例}

100

% 有提示内容的话就去掉下一行行首的注释:符百分号
 \subsection*{提示}
3瓶肥宅快乐水的分配方式如下\newline
1 1 1(三天喝完,一天一瓶)\newline
2 1(两天喝完,第一天两瓶,第二天一瓶)\newline
1 2(两天喝完,第一天一瓶,第二天两瓶)

\end{document} 


% latex数学符号表 https://blog.csdn.net/u013346007/article/details/54138690